\documentclass[12pt]{article}
% Change "article" to "report" to get rid of page number on title page
\usepackage{amsmath,amsfonts,amsthm,amssymb}
\usepackage{setspace}
\usepackage{Tabbing}
\usepackage{fancyhdr}
\usepackage{lastpage}
\usepackage{extramarks}
\usepackage{chngpage}
\usepackage{soul,color}
\usepackage{titlesec}
\usepackage{graphicx,float,wrapfig}

% In case you need to adjust margins:
\topmargin=-0.6in      %
\evensidemargin=0in     %
\oddsidemargin=0in      %
\textwidth=6.5in        %
\textheight=9.0in       %
\headsep=0.25in         %

% Homework Specific Information
\newcommand{\hmwkTitle}{Reading\ \#3}
\newcommand{\hmwkDueDate}{Monday,\ Oct.\ 01,\ 2010}
\newcommand{\hmwkClass}{Cloud\ Computing}
\newcommand{\hmwkAuthorName}{Shumin\ Guo}

% Setup the header and footer
\pagestyle{fancy}                                                       %
\lhead{\hmwkAuthorName}                                                 %
\chead{\hmwkClass\ \hmwkTitle}  %
\rhead{Page\ \thepage\ of\ \pageref{LastPage}} %
\lfoot{}                                                      %
\cfoot{}                                                                %
\rfoot{}                          %
\renewcommand\headrulewidth{0.4pt}                                      %
\renewcommand\footrulewidth{0.4pt}                                      %
\titleformat{\section}{\large\bfseries}{\thesection}{1em}{}
%%%%%%%%%%%%%%%%%%%%%%%%%%%%%%%%%%%%%%%%%%%%%%%%%%%%%%%%%%%%%
% Make title
\title{\textmd{\textbf{\hmwkClass:\
      \hmwkTitle}}\\\normalsize\small{Due\ Date:\
    \hmwkDueDate}\\} 
\date{\today}
\author{\textbf{\hmwkAuthorName}}
%%%%%%%%%%%%%%%%%%%%%%%%%%%%%%%%%%%%%%%%%%%%%%%%%%%%%%%%%%%%%

\begin{document}
\maketitle

%\section*{}
Xen is a x86 virtual machine hypervisor which allows multiple commodity
operating systems to share conventional hardware in a safe and
resource managed way, but without sacrificing either performance or
functionality. It enables users to dynamically instantiate an
operating system to execute whatever they desire. It also provides an
isolated environment for various operating systems to run
simultaneously on one hardware machine. 

Xen uses the paravirtualization approach to multiplexe physical
resources at the granularity of an entire operating system and is able
to provide performance isolation between them. So it allows a range of
guest operating systems to coexist rather than mandating a specific
application binary interface. This feature also enables users to run
unmodified binaries in a resource controlled fashion. It can provide a
very high level of flexibility for users.

Xen is designed in the way similar to a hardware controller for operating
systems that it is hosting. And it exposes a set of clean and simple
device abstractions for the guest operating systems. It sets itself with
highest priviledge and guest operating systems have lower priviledges
so that it can do more priviledged jobs such as instruction
validation, relaying hardware interrupts as a light weight queue
system, physical memory management and creation and termination of an
operating system instance. 

Xen uses hypercall for domains to perform a priviledged operation, and
a asynchronous event system is used for communication from Xen to a
domain. It uttilizes I/O rings for resource management and event
notification. And such communiation mechanisms are used to virtualize
the CPU, Time, Virtual Address translation, Physical memory
allocation, Networking and Disk subsystems. These subsystems are
virtulized such that little efforts are need for guest OSes to be
ported to Xen platform. The task of building the intial guest OS
structures for a new domain is mostly delegated to domain0 which is
the priviledged control interface to access the new Domain's memory
and inform Xen of initial register state.

In real performance evaluations, Xen performs much better than virutal
machines with full virtualization and embedded virutalization and can
achieve identical performance to native operating systems. I think the
reason is that the paravirtualization design places a high emphasis on
performance and resource management. It gracely abstracts the hardware
layer for guest operating systems and the supervises the guest OSes as
a management system. And this hardware level design can provide much
higher efficiency than those embedded or full virutal machines. 

\end{document}