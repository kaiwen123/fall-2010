\documentclass{article}
% Change "article" to "report" to get rid of page number on title page
\usepackage{amsmath,amsfonts,amsthm,amssymb}
\usepackage{setspace}
\usepackage{Tabbing}
\usepackage{fancyhdr}
\usepackage{lastpage}
\usepackage{extramarks}
\usepackage{chngpage}
\usepackage{soul,color}
\usepackage{graphicx,float,wrapfig}
\topmargin=-0.45in      %
\evensidemargin=0in     %
\oddsidemargin=0in      %
\textwidth=6.5in        %
\textheight=9.0in       %
\headsep=0.25in         %

\begin{document}
\begin{center}
\textbf{\textup{\LARGE Cloud Computing Final Exam.}} \\
\textsc{Shumin Guo} \\
\small{Due Date: Nov. 21st, 2010.}
\end{center}

Some of the questions are quite open. Please try your best to give
your answers based on your understanding. You are welcome to search
the web to obtain sufficient background knowledge. Also try to make
your answers concise (less than 1 page per question).   

\begin{enumerate}
\item We have discussed Mapreduce and Pig in the class. Message Passing
Interface (MPI) is another method for parallel computing on
distributed systems, popularly in high performance computing. Please
discuss the advantages and disadvantages of Mapreduce, Pig, and MPI
for solving two types of problems: aggregation style data analysis
jobs and iterative machine learning algorithms, e.g., k-means
clustering. (You can discuss them from different aspects such as disk
I/O, easiness of problem solving and programming, fault tolerance,
etc)


\item Spamming and phishing use relatively low costs by sending emails
that contain false information or advertisements, to take advantage of
careless email receivers. It is challenging to protect internet users
from spamming and phishing. One of the methods is to identify the
spamming and phishing hosts or domains and blacklist them. However,
cloud computing brings particular challenges to this method. Please
use Amazon EC2 to explain why clouds make this method inefficient, and
also give your proposal on detecting and preventing spamming/phishing
in the cloud. 


\item Imaging the scenario that most web services of one company are
deployed in the same public cloud (to make it simpler, we do not
consider multiple clouds). These web services can be non-original,
i.e., mashup of other web services. A simple example is a 3-tier web
application which consists of interacting database server, application
server, and web server. Please describe the unique advantages and
problems for hosting multi-tier web-service applications in the cloud,
compared to the non-cloud solutions (on some aspects such as
performance, resource provisioning, security, etc.).  

\end{enumerate}

\end{document}