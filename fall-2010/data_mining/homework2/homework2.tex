\documentclass{article}
% Change "article" to "report" to get rid of page number on title page
\usepackage{amsmath,amsfonts,amsthm,amssymb}
\usepackage{algorithmic,algorithm}
\usepackage{setspace}
\usepackage{Tabbing}
\usepackage{fancyhdr}
\usepackage{lastpage}
\usepackage{extramarks}
\usepackage{chngpage}
\usepackage{soul,color}
\usepackage{ulem}
\usepackage{graphicx,float,wrapfig}
\usepackage{amsfonts}
\usepackage{pifont}
\usepackage{booktabs}
\usepackage{hyperref}
\usepackage{pstricks,pst-node,pst-tree}
\usepackage{pdftricks}

\newcommand{\tickYes}{\checkmark}
\newcommand{\tickNo}{\hspace{1pt}\ding{55}}

% In case you need to adjust margins:
\topmargin=-0.45in      %
\evensidemargin=0in     %
\oddsidemargin=0in      %
\textwidth=6.8in        %
\textheight=9.4in       %
\headsep=0.25in         %

% Homework Specific Information
\newcommand{\hmwkTitle}{Homework\ \#2}
\newcommand{\hmwkDueDate}{Oct.\ 06,\ 2010}
\newcommand{\hmwkClass}{Data Mining}
\newcommand{\hmwkClassTime}{MW\ 4:10-5:25pm}
\newcommand{\hmwkClassInstructor}{Guozhu\ Dong}
\newcommand{\hmwkAuthorName}{Shumin\ Guo}

% Setup the header and footer
\pagestyle{fancy}                                                       %
\lhead{\hmwkAuthorName}                                                 %
\chead{\hmwkClass\ - \hmwkTitle}  %
\rhead{Page\ \thepage\ of\ \pageref{LastPage}}                          %
\lfoot{\lastxmark}                                                      %
\cfoot{}                                                                %
\rfoot{}                          %
\renewcommand\headrulewidth{0.4pt}                                      %
%\renewcommand\footrulewidth{0.4pt}                                     %

%%%%%%%%%%%%%%%%%%%%%%%%%%%%%%%%%%%%%%%%%%%%%%%%%%%%%%%%%%%%%
% Make title
\title{\textbf{\hmwkClass:\ 
      \hmwkTitle}\\\normalsize\small{Due\ Date:\
    \hmwkDueDate}} 
\date{\today}
\author{\textbf{\hmwkAuthorName}}
%%%%%%%%%%%%%%%%%%%%%%%%%%%%%%%%%%%%%%%%%%%%%%%%%%%%%%%%%%%%%

\begin{document}
% \begin{spacing}{1.1}
\maketitle

\begin{enumerate}
\item Give an example application to discuss how association mining
  can be useful. You should provide a description of the items, how
  relationships between the items can be used for the application
  etc. 

\textbf{ANSWER:}\\
The knowledge provided by association mining can be useful for a lot
of areas, such as e-Marketing, where marketers can promote their
products to customers when they buy some of products. Through
association mining, we can get the association rules between these
products rules. Also, on social networks, systems can try to mining
users' personality or identity and then do related advertisement or
friendship recommendations. \\

\textbf{\underline{For the questions below, consider the following
    example: }} 
\begin{verbatim}
abcd g i
a cd g
 bc  ghi
ab d  hi
 bc eghi
\end{verbatim}

\item For $minSupp=50\%$,apply the APRIORI algorithm to the dataset
  above. You should show work for each level k, show the (a)$C_k$ and
  (b) $L_k$, show (c) the results of the join step and (d) the results
  of the prune step. You should also draw(e) the hash tree. \\
Assume that the items are ordered alphabetically. \\
(f)If you were not able to prune any candidate after the joining step
for all possible k's, try to use a new ordering on the items. If that
still does not let you prune any candidate, try to use a lower
$minSupp$ and perhaps also a new ordering. \\
\textbf{ANSWER:} 
\begin{table}[ht]
  \begin{center}
    \begin{tabular}{r | l}
      \toprule
      Scans & $C_k$(s) and $L_k$(s) \\
      \midrule%
      Scan 1:&a:3 b:4 c:4 d:3 e:1 g:4 h:3 i:3 $\rightarrow C_1$\\
       &a:3 b:4 c:4 d:3 g:4 h:3 i:3 $\rightarrow L_1$ \\
       &ab ac ad ag ah ai bc bd bg bh bi cd cg ch ci dg dh di gh gi hi
      $\rightarrow C_2$ \\
      \midrule
      Scan 2:&ab:2 ac:2 ad:3 ag:2 ah:1 ai:2 bc:3 bd:2 bg:3 bh:3 bi:4
      cd:2 cg:4 ch:2 ci:3 dg:2 dh:1 di:2 gh:2 gi:3 hi:3$\rightarrow C_2$\\
        &ad:3 bc:3 bg:3 bh:3 bi:4 cg:4 ci:3 gi:3 hi:3 $\rightarrow L_2$\\
        &bcg bch bci bgh bgi bhi cgi $\rightarrow C_3$ \\
      \midrule
      Scan 3:&bcg:3 bch:2 bci:3 bgh:2 bgi:3 bhi:3 cgi:3 $\rightarrow
      C_3$ \\
        &bcg:3 bci:3 bgi:3 bhi:3 cgi:3 $\rightarrow L_3$\\
        &bcgi $\rightarrow C_4$ \\
      \midrule
      Scan 4:&bcgi:3 $\rightarrow C_4$ \\
        &bcgi:3 $\rightarrow L_4$~So, we get the $L_4$ frequent
        itemset \textbf{\fbox{bcgi}}.\\
      \bottomrule
    \end{tabular}
    \caption{APRIORI Steps}
  \end{center}
\end{table}
Let's use hash function $h(x)=ord(x)mod3$. And we can get the function
value for all the characters as shown in \ref{tbl:hash_value}
\begin{table}[ht]
  \begin{center}
    \begin{tabular}{|c|c|}
      \toprule Character & hash value \\
      \midrule a & 0 \\
      \midrule b & 1 \\ 
      \midrule c & 2 \\ 
      \midrule d & 0 \\
      \midrule e & 1 \\ 
      \midrule f & 2 \\ 
      \midrule g & 0 \\
      \midrule h & 1 \\ 
      \midrule i & 2 \\ 
      \bottomrule
    \end{tabular}
    \label{tbl:hash_value}
    \caption{Hash Table}
  \end{center}
\end{table}
Please see the tree structure below: \\
\begin{figure}[htb]
  \begin{center}
    \pstree[levelsep=10ex]{\Tcircle{$\Phi$}}{
      \pstree{\Tcircle{a/d/g}\tlput{0} }{ % level - 1 
        \Tcircle{ad}\tlput{0}             % level - 2 
        \Tcircle{gi}\trput{2}
      }
      \pstree{\Tcircle{b/h}\trput{1}}{   % level - 1 
        \pstree{\Tcircle{bg}\tlput{0}} { % level - 2
          \Tcircle{bgi}\tlput{2}
        }
        \pstree{\Tcircle{bh}\tlput{1}} { % level - 2
          \Tcircle{bhi}\tlput{2}
        }
        \pstree{\Tcircle{bc/bi/hi}\tlput{2}} { % level - 2
          \pstree{\Tcircle{bcg}\tlput{0}} { % level - 3
            \Tcircle{bcgi}\tlput{2}         % level - 4
          }
          \Tcircle{bci}\trput{2}
        }
      }
      \pstree{\Tcircle{c/i}\trput{2}}{     %level - 1
        \pstree{\Tcircle{cg}\tlput{0}} { % level - 2
          \Tcircle{cgi}\tlput{2}
        } 
        \Tcircle{ci}\trput{2}
      }
    }
    \caption{Hash Tree For APRIORI}
    \label{tree:hash}
  \end{center}
\end{figure}

\item For the data given above, roughly estimate (A) how many
  candidate itemsets you will need to count if you consider all
  itemsets as candidate itemsets. Also roughly estimate (B) how many
  candidate itemsets you will need to count if you used the "generate
  when seen" approach to candidate generation. (The "generate when
  see" approach counts exactly those itemsets that occur in some
  transactions.) \\
\textbf{ANSWER:} If all the possible itemsets are counted, we need to
count \\
$N = \sum_{k=1}^8C_8^k = C_8^1 + C_8^2 + \ldots + C_8^8 \\
= 8 + 28 + 56 + 70 + 56 + 28 + 8 + 1\\
= 255$ \\

While if we can use the "generate when seen" approach to generate
candidates, we can have: \\
$N = \sum_{k=1}^6C_k \\
= 8 + 21 + 7 + 1 \\
= 37$ \\
As it is clearly can be seen that the last method is far more
efficient than the naive method.

\item 
\begin{enumerate}
\item[(a)] Roughly describe the "hash-based" idea for candidate
  elimination.  \\
\textbf{ANSWER:} When counting candidate(k-1)-itemsets, get counts of
"hash-groups" of k-itemsets.\\
- use a hash function h on k-itemsets \\
- For each transaction t and k-subset s of t, add 1 to count of h(s)
\\
- Remove candidates q generated by Apriori if h(q)'s
count$<$minSupp. \\
- This idea is very useful for $k = 2$, but often not so useful
elsewhere. 

\item[(b)] Roughly describe the partition algorithm for frequent
  itemset mining. \\ 
\textbf{ANSWER:} For large dataset D, it can be divided into smaller
sized partitions $D_1, D_2, \ldots, D_m$ and do association mining to
get feature sets for each partition $F_1, F_2, \ldots, F_m$. If
itemset $X_i$ is frequent in D, it must be frequent in some of $D_i$.
The union of $F_i$ forms a candidate set of the frequent itemsets in
D, and their counts can be used to remove false positives. 

\item[(c)] Is it possible to mine frequent itemsets w/o candidate
  generation? \\
\textbf{ANSWER:} Yes, definitely. 

\item[(d)] From the result of the APRIORI algorithm with
  $minSupp=50\%$, find the corresponding closed itemsets. \\
\textbf{ANSWER:} bi, cg, bcgi. 

\end{enumerate}

\end{enumerate}

\end{document}

%%%%%%%%%%%%%%%%%%%%%%%%%%%%%%%%%%%%%%%%%%%%%%%%%%%%%%%%%%%%%
