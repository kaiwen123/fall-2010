\documentclass{article}
% Change "article" to "report" to get rid of page number on title page
\usepackage{amsmath,amsfonts,amsthm,amssymb}
\usepackage{algorithmic,algorithm}
\usepackage{setspace}
\usepackage{Tabbing}
\usepackage{fancyhdr}
\usepackage{lastpage}
\usepackage{extramarks}
\usepackage{chngpage}
\usepackage{soul,color}
\usepackage{ulem}
\usepackage{graphicx,float,wrapfig}
\usepackage{amsfonts}
\usepackage{pifont}
\usepackage{booktabs}
\usepackage{hyperref}
\usepackage{pstricks,pst-node,pst-tree}

\newcommand{\tickYes}{\checkmark}
\newcommand{\tickNo}{\hspace{1pt}\ding{55}}

% In case you need to adjust margins:
\topmargin=-0.45in      %
\evensidemargin=0in     %
\oddsidemargin=0in      %
\textwidth=6.8in        %
\textheight=9.4in       %
\headsep=0.25in         %

% Homework Specific Information
\newcommand{\hmwkTitle}{Homework\ \#3}
\newcommand{\hmwkDueDate}{Oct.\ 27,\ 2010}
\newcommand{\hmwkClass}{Data Mining}
\newcommand{\hmwkClassTime}{MW\ 4:10-5:25pm}
\newcommand{\hmwkClassInstructor}{Guozhu\ Dong}
\newcommand{\hmwkAuthorName}{Shumin\ Guo}

% Setup the header and footer
\pagestyle{fancy}                                                       %
\lhead{\hmwkAuthorName}                                                 %
\chead{\hmwkClass\ - \hmwkTitle}  %
\rhead{Page\ \thepage\ of\ \pageref{LastPage}}                          %
\lfoot{\lastxmark}                                                      %
\cfoot{}                                                                %
\rfoot{}                          %
\renewcommand\headrulewidth{0.4pt}                                      %
%\renewcommand\footrulewidth{0.4pt}                                     %

%%%%%%%%%%%%%%%%%%%%%%%%%%%%%%%%%%%%%%%%%%%%%%%%%%%%%%%%%%%%%
% Make title
\title{\textbf{\hmwkClass:\ 
      \hmwkTitle}\\\normalsize\small{Due\ Date:\
    \hmwkDueDate}} 
\date{\today}
\author{\textbf{\hmwkAuthorName}}
%%%%%%%%%%%%%%%%%%%%%%%%%%%%%%%%%%%%%%%%%%%%%%%%%%%%%%%%%%%%%

\begin{document}
% \begin{spacing}{1.1}
\maketitle

\begin{enumerate}
\item Compare the advantages and disadvantages of eager classification
  (e.g., decision trees, bayesian, etc.) versus lazy classification
  (e.g., k-nearest neighbors). Eager systems build the model before
  they see the test data, lazy systems do not do model building until
  they see the test data. \\
\textbf{ANSWER:}\\ 
Intuitively, lazy classifiers performs better than eager classifiers
because of two characteristics: 

Missing class association rules(CARs): Eager classifiers search for
CARs in a large search space, which is induced by all features of the
training data. While this strategy generates a large rule-set, CARs
that are important to some specific test instances may be missed. 

And Eager classifiers must commit to a single hypothesis that covers
the entire instance space. It generate CARs before the test instance
is even known. For this reason, eager classifiers often combine small
disjuncts in order to generate more general predicitions. This can
reduce classification performance in highly disjunctive spaces. 

Lazy classifiers generally use less time in training but more time in
predicting Accuracy. Lazy method effectively uses a richer hypothesis
space since it uses many local linear functions to form its implicit
global approximation to the target function. 
However, the same concept also leads to overfitting, reducing the
generalization and affecting the classification accuracy. So,
overfitting and high sensitivity to irrelevant features are
shortcomings of lazy classifications.A natural solution is to identify
and discard the irrelevant features. Thus, feature selection methods
may be used. Another disadvantage is that the lazy classifiers is
typically require more work to classify all test instances. 

\item Why is tree pruning useful in decision tree induction? \\
\textbf{ANSWER:} \\
Pruning is a technique in machine learning that reduces the size of
decision trees by removing sections of the tree that provide little
power to classify instances. The dual goal of pruning is reduced
complexity of the final classifier as well as better predictive
accuracy by the reduction of overfitting and removal of sections of a
classifier that may be based on noisy or erroneous data.

A tree that is too large risks overfitting the training data and
poorly generalizing to new samples. A common strategy is to grow the
tree until each node contains a small number of instances then use
pruning to remove nodes that do not provide additional information.

So, pruned decision trees have several advantages over traditional
decision trees: 
\begin{itemize}
\item They avoid overfitting, because they ignore irrelevant attributes.
\item They are smaller, which makes them easier to understand and more
efficient to store and use. 
\item They can handle noisy data, because they will ignore attributes
rendered meaningless by noise. 
\end{itemize}

\item Given a decision tree, you have the option of (a)converting the
  decision tree to rules and then pruning the resulting rules, or (b)
  pruning the decision tree and then converting the pruned tree to
  rules. Which method is better and why? \\
\textbf{ANSWER:} \\
I will choose method (b).
When a decision tree is built, many of the branches will reflect
anomalies in the training data due to noise or outliers. Tree pruning
methods address this problem of overfitting the data. Pruned trees
tend to be smaller and less complex and, thus, easier to comprehend. 

\item Explain why is the NBC algorithm called naive? \\
\textbf{ANSWER:} \\
A Bayes classifier is a simple probabilistic classifier based on
applying Bayes' theorem (from Bayesian statistics) with strong (naive)
independence assumptions. A more descriptive term for the underlying
probability model would be "independent feature model". 

Given data sets with many attributes, it would be extremely
computationally expensive to compute $P(X|C_i)$. In order to reduce
computation in evaluating $P(X|C_i)$, the naive assumption of class
conditional independence is made. This presumes that the values of the
attributes are conditionally independent of one another, given the
class label of the tuple (i.e., that there are no dependence
relationships among the attributes). 
\begin{equation}
P(X|C_i) = \prod_{k=1}^nP(x_k|C_i) 
= P(x_1|C_i)\times P(x_2|C_i)\times \ldots \times P(x_n|C_i).
\end{equation} 
We can easily estimate the probabilities $P(x_1|C_i), P(x_2|C_i),
\ldots, P(x_n|C_i)$ from the training tuples.

\item The following table consists of training data from an employee
  database.
\begin{table}[H]
  \begin{center}
    \begin{tabular}{|c|c|c|c|}
      \hline {\it department} & {\it status}&{\it age}&{\it salary} \\
      \hline 
      sales & senior & 31-40 & Medium \\
      sales & junior & 21-30 & Low \\
      sales & junior & 31-40 & Low \\
      systems & junior & 21-30 & Medium \\
      systems & senior & 31-40 & High \\
      systems & junior & 21-30 & Medium \\
      systems & senior & 41-50 & High \\
      marketing & senior & 31-40 & Medium \\
      marketing & junior & 31-40 & Medium \\
      secretary & senior & 41-50 & Medium \\
      secretary & junior & 21-30 & Low \\
      \hline
    \end{tabular}
    \caption{Training Data of Employee Data.\label{tbl:dat_eple}}
  \end{center}
  \vspace{-10pt}
\end{table}

\begin{enumerate}
\item[(a)] Let {\it status} be the class attribute. Use the ID3
  algorithm to construct a decision tree from the given data.\\
\textbf{ANSWER:} \\
First, let's decide which attribute will be put on the top of the
tree. We need to calculate the information gain for each attribute
relative the the class {\it status}.

Entropy(S)$ = -P_{junior}log_2(P_{junior})-P_{senior}log_2(P_{senior}) \\
= -\frac{6}{11}log_2\frac{6}{11}-\frac{5}{11}log_2\frac{5}{11} \\
= 0.545455\times 0.87447 + 0.4545455 \times 1.1375035  \\
= 0.47698315 + 0.517047056 \\
\approx~ 0.99403$ \\

Then, Let's calculate the information gain for each attribute: \\
Gain(S, department) $= \\
0.99403-\frac{3}{11}Entropy(S_{sales})-\frac{4}{11}Entropy(S_{systems})-
\frac{2}{11}Entropy(S_{marketing})-\frac{2}{11}Entropy(S_{secretary})\\
= 0.99403 - 0.273\times 0.918 - 0.364\times 1.0 - 0.182\times 1.0 -
0.182\times 1.0 \\
= 0.99403 - 0.2506 - 0.364 - 0.182 - 0.182 \\
\approx~ 0.01543$. \\

Gain(S, age) $ = 
0.99403-\frac{4}{11}Entropy(S_{21-30})-\frac{5}{11}Entropy(S_{31-40})-
\frac{2}{11}Entropy(S_{41-50}) \\
= 0.99403 - 0.364\times 0 - 0.455\times 0.97095 - 0.182\times 0 \\
= 0.99403 - 0.44178 \\
\approx~ 0.55225 $. \\

Gain(S, salary) $ = 
0.99403-\frac{2}{11}Entropy(S_{High})-\frac{6}{11}Entropy(S_{Medium})-
\frac{3}{11}Entropy(S_{Low}) \\
= 0.99403 - 0.182\times 0 - 0.5455\times 1 - 0.273\times 0 \\
= 0.4485 $. \\

Clearly, from the above calculated result, we will choose age as
the first level of tree. And we can get the tree structure as shown in
Figure \ref{fig:id3-1}.
\begin{figure}[H]
  \vspace{-10pt}
  \begin{center}
    \pstree[levelsep=10ex]{\Toval{age}}{
      \Toval{senior}\tlput{41-50}
      \Toval{TBD}\tvput{31-40}
      \Toval{junior}\trput{21-30}
    }
    \caption{First Level Tree Using ID3 Method.\label{fig:id3-1}}
    \vspace{-10pt}
  \end{center}
\end{figure}

After the first level of tree building, we can determine for
senior$\Leftrightarrow$41-50 and junior$\Leftrightarrow$21-30. But for
age group 31-40, it can not be determined using the given data, so
we need to do another step with the similar method as used
above. Please see the reference table for this step at Table
\ref{tbl:medium} The calculations are show as below.\\

\begin{table}[H]
  \begin{center}
    \begin{tabular}{|c|c|c|c|}
      \hline {\it department} & {\it status}&{\it age}&{\it salary} \\
      \hline 
      sales & senior & 31-40 & Medium \\
      sales & junior & 31-40 & Low \\
      systems & senior & 31-40 & High \\
      marketing & senior & 31-40 & Medium \\
      marketing & junior & 31-40 & Medium \\
      \hline
    \end{tabular}
    \caption{Age 31-40 reference table.\label{tbl:medium}}
  \end{center}
  \vspace{-10pt}
\end{table}

Gain($S_{age}$,department)$ = 1.0 - \frac{2}{5}\times 1 -
\frac{1}{5}\times 0 - \frac{2}{5}\times 1 \\
= 1.0 - 0.4 - 0 - 0.4\\
= 0.20$. \\

Gain($S_{age}$,salary)$ 
= 1.0-\frac{1}{5}\times 0-\frac{1}{5}\times 0 -\frac{3}{5}\times 0.9183\\
= 1.0 - 0 - 0 - 0.5510\\
= 0.4490$. \\

From the above calculation, we will select {\it salary} as the
attribute for next level decision tree construction. And we can get
the following tree as show in Figure \ref{fig:id3-2}.\\

\begin{figure}[H]
  \vspace{-10pt}
  \begin{center}
    \pstree[levelsep=10ex]{\Toval{age}}{
      \Toval{senior}\tlput{41-50}
      \pstree{\Toval{salary}\tvput{31-40}} {
        \Toval{junior}\tlput{Low}
        \Toval{senior}\tvput{Medium}
        \Toval{senior}\trput{High}
      }
      \Toval{junior}\trput{21-30}
    }
    \caption{Second Level Tree Using ID3 Method.\label{fig:id3-2}}
    \vspace{-10pt}
  \end{center}
\end{figure}

Figure \ref{fig:id3-2} is the final result of the decision tree building. 

\item[(b)] Given an instance with the values systems, senior, and
  21-30 for the attribute {\it department},{\it status}, and
  {\it age}, respectively, what would be a naive bayesian
  classification for the {\it salary} of the sample be? You want to
  use "Laplace correction" for probability estimates. Show all the
  steps. \\
\textbf{ANSWER:} \\
For the class {\it salary}, we have three possible values, Low, Medium
and High respectively.

According to the requirement, the tuple that we need to classify is:
X=(department=systems, status=senior, age=21-30) \\
We need to maximize $P(X|C_i)P(C_i)$, for $i=1,2,3$. let's first
compute the prior probability of each class: \\
P(salary=Low) = 3/11 = 0.273 \\
P(salary=Medium) = 6/11 = 0.5455\\
P(salary=High) = 2/11 = 0.182\\

Then, let's compute the conditional probalities: \\
P(department=systems $|$ salary=Low) = 0.0 \\
P(department=systems $|$ salary=Medium) = 2/4 = 0.5 \\
P(department=systems $|$ salary=High) = 2/4 = 0.5 \\
P(status=senior $|$ salary=Low) = 0.0 \\
P(status=senior $|$ salary=Medium) = 2/4 = 0.5 \\
P(status=senior $|$ salary=High) = 2/4 = 0.5 \\
P(age=21-30 $|$ salary=Low) = 2/4 = 0.5 \\
P(age=21-30 $|$ salary=Medium) = 2/4 = 0.5 \\
P(age=21-30 $|$ salary=High) = 0.0 \\

With the above possibilities, we can have: \\ 
P(X$|$salary=Low) = P(department=systems$|$salary=Low)$\times$ \\
P(status=senior$|$salary=Low)$\times$ P(age=21-30$|$salary=Low) \\
= $0.0 \times 0.0 \times 0.5$ \\
= 0.0 \\

Similarly, we have: \\
P(X$|$salary=Medium) = $0.5 \times 0.5 \times 0.5 = 0.125$. \\
P(X$|$salary=High) = $0.5 \times 0.5 \times 0.0 = 0.0$. \\

Then, let's compute $P(X|C_i)P(C_i)$, and maximize it: \\
P(X$|$salary=Low)$\times$P(salary=Low) = 0.0 \\
P(X$|$salary=Medium)$\times$P(salary=Medium) \\
= 0.125$\times$0.5455 = 0.0682 \\
P(X$|$salary=High)$\times$P(salary=High) = 0.0

Therefore, the naive Bayesian classfier predicts salary=Medium for
tuple X. \\

\textbf{With Laplacian correction}, we can avoid computing probability
values of zero as what has happened above. \\
First, we have the conditional probability as below: \\
P(department=systems $|$ salary=Low) = 1/7 = 0.143 \\
P(department=systems $|$ salary=Medium) = 3/7 = 0.429 \\
P(department=systems $|$ salary=High) = 3/7 = 0.429 \\
P(status=senior $|$ salary=Low) = 1/7 = 0.143 \\
P(status=senior $|$ salary=Medium) = 3/7 = 0.429 \\
P(status=senior $|$ salary=High) = 3/7 = 0.429 \\
P(age=21-30 $|$ salary=Low) = 3/7 = 0.429 \\
P(age=21-30 $|$ salary=Medium) = 3/7 = 0.429 \\
P(age=21-30 $|$ salary=High) = 1/7 = 0.143\\

P(X$|$salary=Low) = P(department=systems$|$salary=Low)$\times$ \\
P(status=senior$|$salary=Low)$\times$ P(age=21-30$|$salary=Low) \\
= $0.143 \times 0.143 \times 0.429$ \\
$\approx$~ 0.008773\\

Similarly, we have: \\
P(X$|$salary=Medium) = $0.429 \times 0.429 \times 0.429 = 0.078954$. \\
P(X$|$salary=High) = $0.429 \times 0.429 \times 0.143 = 0.02632$. \\

Similarly, we have: \\
P(X$|$salary=Low)$\times$P(salary=Low)=0.008773$\times$0.273=0.002395\\
P(X$|$salary=Medium)$\times$P(salary=Medium) \\
= 0.078954$\times$0.5455 = 0.04307 \\
P(X$|$salary=High)$\times$P(salary=High) = 0.02632$\times$0.182 =
0.00478. \\
Still, we have NBC predicts salary = Medium for tuple X.
\end{enumerate} % For question five. 

\item Given the data points (0,0/1), (1,0/1), (1,1/1), (4,3/2),
  (3,4/2), (1,4/2) where (x,y/c) gives a sample with x and y as
  attribute values and c the class label. Use k-NN to find the class
  label for (2,2/?) when k=1, and when k=5. Use Euclidean distance as
  the distance measure. \\ 
\textbf{ANSWER:} \\
Let's compute the distance for (2,2) to each point. \\
(0,0/1) and (2,2/?) $\rightarrow$ $\sqrt{8}$ = 2.828\\
(1,0/1) and (2,2/?) $\rightarrow$ $\sqrt{5}$ = 2.236\\
(1,1/1) and (2,2/?) $\rightarrow$ $\sqrt{2}$ = 1.414\\ 
(4,3/2) and (2,2/?) $\rightarrow$ $\sqrt{5}$ = 2.236\\
(3,4/2) and (2,2/?) $\rightarrow$ $\sqrt{5}$ = 2.236\\
(1,4/2) and (2,2/?) $\rightarrow$ $\sqrt{5}$ = 2.236\\

When k=1, (1,1/1) has the nearest Euclidean distance with (2,2/?), so
we have the result: (2,2/1).\\

When k=5, we have two samples with class 1 and three samples with
class 2, so we have the result: (2,2/2).


\end{enumerate} % For whole homework. 

\end{document}

%%%%%%%%%%%%%%%%%%%%%%%%%%%%%%%%%%%%%%%%%%%%%%%%%%%%%%%%%%%%%
