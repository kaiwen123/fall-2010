\documentclass{article}
\usepackage{verbatim}

\begin{document}
\begin{center}
\textbf{{\Large CS 400/600 – Data Structures and Software Design} \\
{\large Programming Assignment \#4 –Hash Table}}
\end{center}

\section*{Report}
Along with your working source code (.cpp and .h files), you should
submit a brief ($<$ 5 pages) report describing how the closed hash
function performed as a function of the load factor of the hash table,
alpha. You should perform a variety of searches, and compare the
number of nodes searched for each index.  You should address questions
such as: 
\begin{itemize}
\item How does the hash perform in terms of search efficiency when the hash
  table is (1) nearly empty (the alpha is very small, like less than
  0.1),  (2) moderately full (alpha is around 0.5),  (3) very full
  (alpha is larger than 0.8). 
\item Give some intuitive comments about the performance of your hash table
  regarding the load factor, alpha.  
\end{itemize}

Your report should be supported with data in the form of tables,
charts, graphs, etc. 

Make sure to base your report on average results over many searches,
and not just one search. 

\section*{Graduate Students}
CS 600 students should evaluate an additional hashing method (other
than closed double-hashing), and include these results in the final
report (for example, you might choose to investigate the performance
of a closed hash using pseudo-random hashing. You can also implement a
chained hash table if you like). 

\end{document}

%%%%%%%%%%%%%%%%%%%%%%%%%%%%%%%%%%%%%%%%%%%%%%%%%%%%%%%%%%%%%
