\documentclass{article}
% Change "article" to "report" to get rid of page number on title page
\usepackage{amsmath,amsfonts,amsthm,amssymb}
\usepackage{algorithmic,algorithm}
\usepackage{setspace}
\usepackage{Tabbing}
\usepackage{fancyhdr}
\usepackage{lastpage}
\usepackage{extramarks}
\usepackage{chngpage}
\usepackage{soul,color}
\usepackage{ulem}
\usepackage{graphicx,float,wrapfig}
\usepackage{pifont}
\usepackage{booktabs}
\usepackage{hyperref}
\usepackage{pstricks,pst-node,pst-tree}
\usepackage{pdftricks}
\usepackage{subfigure}
\usepackage{multicol}
\usepackage{enumerate}

\usepackage{listings}
\usepackage{color}
\usepackage{textcomp}
\definecolor{listinggray}{gray}{0.9}
\definecolor{lbcolor}{rgb}{0.9,0.9,0.9}
\lstset{
	%backgroundcolor=\color{lbcolor},
	tabsize=4,
	rulecolor=,
	language=c++,
        basicstyle=\setstretch{1},
        upquote=true,
        aboveskip={\baselineskip},
        columns=fixed,
        showstringspaces=false,
        extendedchars=true,
        breaklines=true,
        prebreak = \raisebox{0ex}[0ex][0ex]{\ensuremath{\hookleftarrow}},
        %frame=single,
        showtabs=false,
        showspaces=false,
        showstringspaces=false,
        identifierstyle=\ttfamily,
        keywordstyle=\color[rgb]{0,0,1},
        commentstyle=\color[rgb]{0.133,0.545,0.133},
        stringstyle=\color[rgb]{0.627,0.126,0.941},
}
% In case you need to adjust margins:
\topmargin=-0.45in      %
\evensidemargin=0.5in     %
\oddsidemargin=0.5in      %
\textwidth=6.0in        %
\textheight=9.0in       %
\headsep=0.25in         %

% Homework Specific Information
\newcommand{\hmwkTitle}{Programming Assignment\ \#4}
\newcommand{\hmwkDueDate}{Nov.\ 11th,\ 2010\ 11:55pm}
\newcommand{\hmwkClass}{Data Structure}
\newcommand{\hmwkClassTime}{TR\ 4:10-5:25pm}
\newcommand{\hmwkClassInstructor}{Meilin\ Liu}
\newcommand{\hmwkAuthorName}{Shumin\ Guo}
\newcommand{\answer}{\textbf{\\\underline{ANSWER:}\\}}

% Setup the header and footer
\pagestyle{fancy}                                                       %
\lhead{\hmwkAuthorName}                                                 %
\chead{\hmwkClass\ - \hmwkTitle}  %
\rhead{Page\ \thepage\ of\ \pageref{LastPage}}                          %
\lfoot{\lastxmark}                                                      %
\cfoot{}                                                                %
\rfoot{}                          %
\renewcommand\headrulewidth{0.4pt}                                      %
%\renewcommand\footrulewidth{0.2pt}                                     %

%%%%%%%%%%%%%%%%%%%%%%%%%%%%%%%%%%%%%%%%%%%%%%%%%%%%%%%%%%%%%
% Make title
\title{\textmd{\textbf{\hmwkClass\\\
      \hmwkTitle}}\\\normalsize\small{Due\ Date:\
    \hmwkDueDate}\\} 
\date{\today}
\author{\textsc{\hmwkAuthorName}}
%%%%%%%%%%%%%%%%%%%%%%%%%%%%%%%%%%%%%%%%%%%%%%%%%%%%%%%%%%%%%

\begin{document}
\begin{center}
\textbf{{\Large CS 400/600 – Data Structures and Software Design} \\
{\large Programming Assignment \#4 –Hash Table}}
\end{center}

\section*{Overview}
The objective of this assignment will be to implement a closed-double
hash, and submit a report that gives  discussion of the result that
you get based on the load factor of the hash table, alpha.  

\section*{Programming}
For the programming section, you will create a closed double hash to
index positive integer values (employee ID’s) between 1 and 999,999.
You should create a hash of size N = 32,768 (32k) slots. 

You will be provided a C++ abstract class called Hash, you should
construct class ClosedHash as a subclass of class Hash, as follows:
\begin{verbatim}
  class ClosedHash : publich Hash { … }; 
\end{verbatim}
Your hash should use the EMPTY constant to mark empty hash buckets,
and should use and fully support tombstones (using the TOMBSTONE
constant) for deletions. 

You will also be provided with an extensive main program for testing
your hash implementation. The primary hash function $h_1(k)$ and the
secondary hash function $h_2(k)$ are also given. The primary hash
function gives the home location for a particular key. The secondary
hash function determines the probe offset for a particular key. 

\section*{Report}
Along with your working source code (.cpp and .h files), you should
submit a brief ($<$ 5 pages) report describing how the closed hash
function performed as a function of the load factor of the hash table,
alpha. You should perform a variety of searches, and compare the
number of nodes searched for each index.  You should address questions
such as: 
\begin{itemize}
\item How does the hash perform in terms of search efficiency when the hash
  table is (1) nearly empty (the alpha is very small, like less than
  0.1),  (2) moderately full (alpha is around 0.5),  (3) very full
  (alpha is larger than 0.8). 
\item Give some intuitive comments about the performance of your hash table
  regarding the load factor, alpha.  
\end{itemize}

Your report should be supported with data in the form of tables,
charts, graphs, etc. 

Make sure to base your report on average results over many searches,
and not just one search. 

The preferred format for your report is a Microsoft Word file named
proj4-report.doc, included in your project 4 directory and turned in
with your source code.  Other formats (like pdf format) may also be
acceptable, as long as the report is clear. 

\section*{Requirements}
\begin{enumerate}
\item Your code should follow the Code Standards handed out during the
  first day of class.  Your code will be graded according to its
  correctness, efficiency, organization, and readability. 
\item Make sure that each file includes your name in the header
  comments.
\item Turn in all files needed to compile and execute your code
  (\underline{including any needed files from the previous lab}) via
  webCT. If for some reason WebCT is unavailable, submit your source
  code by email to wlodarski.4 AT wright.edu. If you want, you can
  also cc to the instructor Meilin Liu, whose email address is
  meilin.liu AT wright.edu.  
\item If there are any special instructions that I need to know about,
  be sure to include a file named Readme.txt in your project 3
  directory detailing them. 
\item The grader will test your programs under the schools UNIX
  environment, e.g., unixapps1.wright.edu. It is YOUR responsibility
  to make your programs workable and runnable by others under school’s
  UNIX environment. 
\item The programming assignment is individual. You must do the
  project by yourself. If you allow others to copy your programs or
  answers, you will get the same punishment as those who copy yours. 
\end{enumerate}

\section*{Graduate Students}
CS 600 students should evaluate an additional hashing method (other
than closed double-hashing), and include these results in the final
report (for example, you might choose to investigate the performance
of a closed hash using pseudo-random hashing. You can also implement a
chained hash table if you like). 

\end{document}

%%%%%%%%%%%%%%%%%%%%%%%%%%%%%%%%%%%%%%%%%%%%%%%%%%%%%%%%%%%%%
