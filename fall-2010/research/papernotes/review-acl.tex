\documentclass[24pt]{article}
% Change "article" to "report" to get rid of page number on title page
\usepackage{amsmath,amsfonts,amsthm,amssymb}
\usepackage{setspace}
\usepackage{Tabbing}
\usepackage{fancyhdr}
\usepackage{lastpage}
\usepackage{extramarks}
\usepackage{chngpage}
\usepackage{soul,color}
\usepackage{titlesec}
\usepackage{graphicx,float,wrapfig}

% In case you need to adjust margins:
\topmargin=-0.45in      %
\evensidemargin=0in     %
\oddsidemargin=0in      %
\textwidth=6.5in        %
\textheight=9.0in       %
\headsep=0.25in         %

% Homework Specific Information
\newcommand{\hmwkTitle}{Reading\ \#2}
\newcommand{\hmwkDueDate}{Monday,\ Oct.\ 01,\ 2010}
\newcommand{\hmwkClass}{Cloud\ Computing}
\newcommand{\hmwkAuthorName}{Shumin\ Guo}

% Setup the header and footer
% \pagestyle{fancy}                                                       %
% \lhead{\hmwkAuthorName}                                                 %
% \chead{\hmwkClass\ \hmwkTitle}  %
% \rhead{Page\ \thepage\ of\ \pageref{LastPage}} %
% \lfoot{}                                                      %
% \cfoot{}                                                                %
% \rfoot{}   
% 
% \renewcommand\headrulewidth{0.4pt}                                      %
% \renewcommand\footrulewidth{0.4pt}                                      %
\titleformat{\section}{\large\bfseries}{\thesection}{1em}{}
%%%%%%%%%%%%%%%%%%%%%%%%%%%%%%%%%%%%%%%%%%%%%%%%%%%%%%%%%%%%%
% Make title
\title{\textmd{\textbf{\hmwkClass:\
      \hmwkTitle}}\\\normalsize\small{Due\ Date:\
    \hmwkDueDate}\\} 
\date{\today}
\author{\textbf{\hmwkAuthorName}}
%%%%%%%%%%%%%%%%%%%%%%%%%%%%%%%%%%%%%%%%%%%%%%%%%%%%%%%%%%%%%

\begin{document}

\section*{Known Access Control Privacy Protection Issues.}
Online social networks provide a platform for people to publish
details about themselves and to connect to other members of the
network through various relationships. These services provide a lot
of benefits to people within this virtual world. But the
huge amount of data including person specific information creates both
interesting research challenges and security threats. \\
While the privacy issue has raised a lot of concerns among OSN users,
current OSNs implement very basic access control models. Although
easy to use and understand, these models lack flexibility and often
times fail to protect the privacy of users. So, how to protect the
privacy of OSN users have become a very interesting and hot research
topic. \\
Social Networks are built upon relations between users, so the access
control and privacy protection models should be built based on the
relationships among users. A lot of research proposals expressed the
access control requirements in terms of relationship paths existing in
the network and their depth. Also, some models support a notion of
trust/reputation as a further parameter for access control decisions.

Besides, the enforcement of relationship-based access control poses
interesting issues regarding privacy protection. A further issue is
the architecture on the support of access control. Unlike traditional
access control models such as rule based access control, the
centralized access control are no longer a suitable or efficient model
for OSNs. A decentralized privacy-aware access control mechanism
should be devised to enforce not only relationship-based access
control but also ensure the relationship privacy. 

\section*{Access Control Requirements.}
According to a relationship-based access control model, access control
policies are specified in terms of relationships existing in the
OSN. Additionally, the depth of the relationship path is an important
parameter for some access control decisions, since users are usually
more inclined to share their resources with users not much away from
them. Still a further important parameter is represented by trust
among social network nodes. \\
Moreover, in OSN, there exists relationship between users and
resources. So, a relationship-based access control model should
exploit not only the standard user to user relationships but should
also consider the various relationships and connection between users
and resources. \\
As we have discussed earlier, a centralized access control model is
not suitable for OSNs, a decentralized model should be used, where
each user is responsible for policy specification and enforcement. But
the decentralized model need to verify the existence of specific paths
within an OSN. This kind of task may be very difficult and time
consuming in a decentralized manner. So, a further essential
requirement of access control enforcement is to devise efficient and
scalable implementation strategies. \\
On the other hand, the relationship-based access control also poses
threats to the relationships which are in general sensitive
information. As the relationship-based access control may require
disclosure of personal relationships. So, another requirement for
relationship-based access control model is to ensure that relationship
privacy is not breached during access control. Also, relationships may
have an associated trust value that must be protected during access
control. So, there will be a requirement to protect the relationship
trust level.\\
Also, the decentralized enforcement of access control should have some
mechanism to help a user to precisely estimate the other users's trust
level. And such mechanism should also preserve user privacy when
performing trust computation. 

\section*{Some Review of Literature.}
In current available literatures, various access control models have
been proposed, such as access rule based model, client-based access
control model, ACL model, multi-level access control,
challenge-response-based model etc. 
Also, there are proposals on distance based access control, in which
requestors are divided into adjacent zones, and different zones have
different rules or policies for access control. 
A lot of literature also proposed cryptographic methods which are used
for relationship certificate exchange, attestation and so on. And there
are also a formalized access control model proposed in the literature,
in which the authers proposed several types of access control
policies, resources access control, search policy, tranversal policies
and communication policies. And recent research on semantic web also
provides opportunities for access control research. \\

But, it is important to know that most of the research related to
privacy in OSNs have focused on privacy-preserving techniques to mine
social network data, only a few provide solutions for privacy aware
access control. Among these privacy aware access control models, some
are policy based, while others consider trust protection. 

\section*{Interesting research topics.}
Research in OSN privacy related area is still in its infancy, there
are a lot of issues to be explored. One is the design of a
satisfactory solution to relationship privacy protection during path
discovery. Another research direction is related to policy
administration. Due to the large size of social network and relations,
it is important to devise techniques and tools that can help user
evaluate the risk of unauthorized flows of information that the
specification of a policy or its update may cause.

\end{document}