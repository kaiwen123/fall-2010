\documentclass{article}
% Change "article" to "report" to get rid of page number on title page
\usepackage{amsmath,amsfonts,amsthm,amssymb}
\usepackage{algorithmic,algorithm}
\usepackage{setspace}
\usepackage{Tabbing}
\usepackage{fancyhdr}
\usepackage{lastpage}
\usepackage{extramarks}
\usepackage{chngpage}
\usepackage{soul,color}
\usepackage{ulem}
\usepackage{graphicx,float,wrapfig}
\usepackage{pifont}
\usepackage{booktabs}
\usepackage{hyperref}
\usepackage{pstricks,pst-node,pst-tree}
\usepackage{pdftricks}
\usepackage{subfigure}
\usepackage{multicol}
\usepackage{enumerate}
\usepackage{pifont}
\usepackage{listings}
\usepackage{color}
\usepackage{textcomp}
\newcommand{\tickYes}{\checkmark}
\newcommand{\tickNo}{\hspace{1pt}\ding{55}}

\definecolor{listinggray}{gray}{0.9}
\definecolor{lbcolor}{rgb}{0.9,0.9,0.9}

\lstset{
	%backgroundcolor=\color{lbcolor},
	tabsize=4,
	rulecolor=,
	language=c++,
        basicstyle=\setstretch{1},
        upquote=true,
        aboveskip={\baselineskip},
        columns=fixed,
        showstringspaces=false,
        extendedchars=true,
        breaklines=true,
        prebreak = \raisebox{0ex}[0ex][0ex]{\ensuremath{\hookleftarrow}},
        %frame=single,
        showtabs=false,
        showspaces=false,
        showstringspaces=false,
        identifierstyle=\ttfamily,
        keywordstyle=\color[rgb]{0,0,1},
        commentstyle=\color[rgb]{0.133,0.545,0.133},
        stringstyle=\color[rgb]{0.627,0.126,0.941},
}
% In case you need to adjust margins:
\topmargin=-0.45in      %
\evensidemargin=0in     %
\oddsidemargin=0in      %
\textwidth=7.0in        %
\textheight=9.2in       %
\headsep=0.25in         %

% Setup the header and footer
% \pagestyle{fancy}                                                       %
% \lhead{\hmwkAuthorName}                                                 %
% \chead{\hmwkClass\ - \hmwkTitle}  %
% \rhead{Page\ \thepage\ of\ \pageref{LastPage}}                          %
% \lfoot{\lastxmark}                                                      %
% \cfoot{}                                                                %
% \rfoot{}                           %
\renewcommand\headrulewidth{0.4pt}                                      %

%%%%%%%%%%%%%%%%%%%%%%%%%%%%%%%%%%%%%%%%%%%%%%%%%%%%%%%%%%%%%

\begin{document}
\begin{center}
\textbf{\Huge{CS 740 Execise Set 1}}\\
\textsc{Shumin Guo}
\end{center}

\begin{enumerate}
\LARGE{\item Show: $2^n \in O(n!)$\label{exe:1}}
\large{
\begin{table}[ht]
  \begin{center}
    \begin{tabular}{cccc}
      \toprule $n$ & $2^n$ & $n!$ & $(2^n < n!) ?$ \\
      \midrule 1 & 2 & 1 & \tickNo \\
      \midrule 2 & 4 & 2 & \tickNo \\
      \midrule 3 & 8 & 6 & \tickNo \\
      \midrule 4 & 16 & 24 & \tickYes \\
      \midrule 5 & 32 & 120 & \tickYes \\
      \bottomrule
    \end{tabular}
    \caption{Enumerating function values for $n\in[1,4]$\label{enum}}
  \end{center}
\end{table}

With the enumeration in Table \ref{enum}, let's assume $C=1$ and
$n_0=4$, s.t. \\ $2^n \le C.n! = n!$ for all $n\ge n_0 = 4$. 

\begin{proof}
(With mathematical induction) \\
\ding{172} When $n = 4$, we have $2^4 = 16$ and $4! = 24$, so $2^n\le 
n!$ holds. \\  
\ding{173} Assume $2^n \le n!$ holds for $n = k$ where $k > 4$. \\
Then, when $n = k + 1$, we have $2^{k+1}=2.2^k\le 2.k!\le (k+1).k! = (k+1)!$, \\
which means $2^{k+1} \le (k+1)!$ \\
$\therefore$ $2^n \le n!$ for all $n\ge 4$\\
$\therefore$ $2^n\in O(n!)$ 
\end{proof}
}
\LARGE{\item Show: $n! \notin O(2^n)$}

\large{
  Assume $n! \in O(2^n)$, we have $\exists C_0$ and $n_0$ s.t. $n!\le C_0.2^n$
  for all $n\ge n_0$. \\ 
  And according to Execise \ref{exe:1}, we have $\exists C_1$ and $n_1$
  s.t. $2^n\le C_1.n!$ for all $n\ge n_1$.  \\
  Let $C_1 = C_0$, we can compute $n_0\le n_1$$<+\infty$ such that
  $2^n\le C_1.n!$ for all $n\ge n_1$, which controdicts the
  assumption. \\  
  $\therefore n! \notin O(2^n)$
}

\LARGE{\item Show: if $f\in O(g)$ and $g\in O(h)$, then $f\in O(h)$}

\large{
  $\because f\in O(g)$ \\
  $\therefore \exists$ $C$ and $n_0$ s.t. $f(n)\le C.g(n)$ for all
  $n\ge n_0$ \\ 
  And 
  $\because g\in O(h)$ \\
  $\therefore \exists$ $C^{\prime}$ and $n_0^{\prime}$ s.t. $g(n)\le
  C^{\prime}h(n)$ for 
  all $n\ge n_0^{\prime}$. \\ 
  $\Rightarrow C.g(n) \le C.C^{\prime}.h(n)$ for all $n\ge
  n_0^{\prime}$. \\ 

  Let $C_1$ = $C.C^{\prime}$ and $n_1$ = max\{$n_0, n_0^{\prime}$\}, we have 
  $f(n) \le C_1.h(n)$ for all $n\ge n_1$. \\ 
  $\therefore f\in O(h)$
}
\end{enumerate}
\end{document}