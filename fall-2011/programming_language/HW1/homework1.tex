\documentclass{article}
\usepackage{graphicx}
\usepackage{setspace}
\usepackage{anysize}
\marginsize{1cm}{1cm}{0cm}{3cm}

\fontfamily{ptm}\fontsize{14}{14}\selectfont 
\onehalfspacing

\setlength{\parindent}{0pt}
% \setlength{\textwidth}{1.5\textwidth}
\newcommand{\ttt}[1]{\ttfamily\selectfont{#1}\normalfont}
\begin{document}
\begin{center}
\textbf{\large{
Programming Language (CS784) \\
Homework 1 \\
(Due: Friday, Sep. $23^{th}$, 2011) \\}}
\textsc{\Large{Shumin Guo}}
\end{center}

\begin{enumerate}
  \item (10 points) EOPL3 Exercise 1.12. Eliminate the one call to
    \ttt{subst-in-s-exp} in subst by replacing it by its definition and
    simplifying the resulting procedure. The result will be a version
    of subst that does not need \ttt{subst-in-s-exp}. 
\item (10 points) EOPL3 1.20.  \ttt{(count-occurrences s slist)} returns the
  number of occurrences of s in slist.
\item (10 points) EOPL3 Exercise 1.28. \ttt{(merge lon1 lon2)}, where \ttt{lon1}
  and \ttt{lon2} are lists of numbers that are sorted in ascending order,
  returns a sorted list of all the numbers in \ttt{lon1} and \ttt{lon2}. 
\item (20 points) Define the procedure compose such that \ttt{(compose p1
  p2)}, where \ttt{p1} and \ttt{p2} are procedures of one argument, returns the
  composition of these procedures, specified by this equation: \\
  \ttt{
  ((compose p1 p2) x) = (p1 (p2 x)) \\
  > ((compose car cdr) '(a b c d)) \\
  b}
\item (20 points) \ttt{(car\&cdr s slist errvalue)} returns an expression
  that, when evaluated, produces the code for a procedure that takes a
  list with the same structure as slist and returns the value in the
  same position as the leftmost occurrence of s in slist. If s does
  not occur in slist, then errvalue is returned. Do this so that it
  generates procedure compositions. \\
  \ttt{
  > (car\&cdr 'a '(a b c) 'fail) \\ 
  car \\
  > (car\&cdr 'c '(a b c) 'fail) \\ 
  (compose car (compose cdr cdr))  \\
  > (car\&cdr 'dog '(cat lion (fish dog ()) pig) 'fail) \\ 
  (compose car (compose cdr (compose car (compose cdr cdr)))) \\ 
  > (car\&cdr 'a '(b c) 'fail) \\
  fail }

\item (30 points) EOPL3 Exercise 2.30 The procedure parse-expression
  as defined p53 is fragile: it does not detect several possible
  syntactic errors, such as \ttt{(a b c)}, and aborts with
  inappropriate error messages for other expressions, such as
  \ttt{(lambda)}. Modify it so that it is robust, accepting any
  \ttt{s-exp} and issuing an appropriate error message if the
  \ttt{s-exp} does not represent a \ttt{lambda-calculus} expression.

\end{enumerate}

\end{document}